\section{Introduction}

The Large Hadron Collider beauty (LHCb) experiment at CERN is a general purpose detector in the forward region, which focuses on investigating the differences between matter and antimatter by studying the decays of beauty (\emph{B}) and charm (\emph{D}) mesons. The detector has been recording data from proton-proton collisions since 2010 and is expected to record data throughout the 2020s. Due to the rapid development of both the hardware and software used to process the data, many questions have been raised about data compatibility and preservation. In response to this, we are creating a database to record the metadata of our software and the data provenance. The recorded dependencies are expected to ease the process of running the software and analysing the data in the long-term future.

\subsection {Data preservation initiative}

 The data preservation project in High Energy Physics (HEP) aims to ensure the preservation of experimental and simulated data, as well as the scientific software and documentation. The main objective is to assist analysing HEP data in the future. The major use cases include looking for signals predicted by new theories and improving current measurements, in addition to physics outreach and educational purposes.

The LHCb data is processed with software and hardware that are changing over time. The information about the data, software and the changes have been logged in the internal databases and web portals. Our goal was to collect and structure this information into a singular, robust database that can be used immediately for scientific purposes and for the long-term future preservation. 

Finally, in support of making the LHCb data replicable and reusable, we concentrate our efforts to provide complete information on the data provenance and, to some extent, assist in its recreation independently of the CERN infrastructure.
